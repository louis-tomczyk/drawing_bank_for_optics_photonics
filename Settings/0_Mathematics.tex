% ------------------------------------------------------------------------------
% ----- INFORMATIONS -----
%   Author          : louis tomczyk
%   Institution     : Telecom Paris
%   Email           : louis.tomczyk@telecom-paris.fr
%   Arxivs          : 2024-05-01 (1.0.0)
%                   : 2024-05-06 [NEW] pipe, mymax, myargmax, integers
%                   : 2024-05-25 [NEW] myeq
%   Date            : 2024-05-30 [NEW] sgn
%   Version         : 1.1.0
%   License         : cc-by-nc-sa
%                       CAN:    modify - distribute
%                       CANNOT: commercial use
%                       MUST:   share alike - include license
% ------------------------------------------------------------------------------

% ============================================================== DEFINITIONS
% ----------- A
\def \And{\quad\&\quad}

% ----------- B
% ----------- C
\def \calEH{\mathcal{E}^\mathtt{H}}
\def \calET{\mathcal{E}^\mathtt{T}}
\def \calXhat{\hat{\mathcal{X}}}
\def \card{\mathrm{card}}
\def \cov{\text{cov}}

% ----------- D
\def \d{\cdot}
\def \DAPR{\mathrm{DAPR}}
\def \dPhiPol{\Delta\Phi_{pol}}

% ----------- E
\def \epso{\varepsilon_0}
\def \erf{\mathrm{erf}}
\def \erfc{\mathrm{erfc}}

% ----------- F
\def \fdTheta{f_{\Delta\theta}}
\def \fdTheta{f_{\Delta\theta}}
\def \fdThetacorr{f_{\Delta\theta,corr}}


% ----------- G
% ----------- H
\def \hatt{h_{att}}
\def \hdc{h_{DC}}
\def \hhat{\hat{h}}
\def \Hhat{\hat{H}}
\def \hlin{h_{lin}}
\def \hlm{\vec{h}_{\lambda\mu}}
\def \hlmI{\vec{h}_{\lambda\mu,I}}
\def \hlmQ{\vec{h}_{\lambda\mu,Q}}
\def \hnlin{h_{nlin}}
\def \hpdl{h_{PDL}}
\def \hpmd{h_{PMD}}

% ----------- I
\def \ind{\mathbb{1}}
\def \Id{\bm{I_d}}

% ----------- J
% ----------- K
% ----------- L
\def \l{\left}
\def \Lbp{L_{bp}}
\def \Lcorr{L_{corr}}


% ----------- M
% ----------- N
% ----------- O
% ----------- P
\def \pxy{p(x|y)}
\def \pyx{p(y|x)}
\def \px{p(x)}
\def \py{p(y)}
\def \pipe{\text{\textbar}}

% ----------- Q
\def \qxy{q(x|y)}
\def \qyx{q(y|x)}
\def \qx{q(x)}
\def \qy{q(y)}


% ----------- R
\def \r{\right}
\def \Rsymb{R_{symb}}
\def \Rbit{R_{bit}}


% ----------- S
\def \Sp{\mathrm{Sp}}
\def \sgn{\mathrm{sgn}}

% ----------- T
\def \tq{\quad \backslash\quad}
\def \Tpolcorr{T_{\text{pol,corr}}}
\def \Tpolbp{T_{\text{pol,bp}}}
\def \Tsymb{T_{symb}}
\def \Tbit{T_{bit}}
\def \Tr{\mathrm{Tr}}


% ----------- U
% ----------- V
\def \vbeta{\vec{\beta}}
\def \vcalB{\vec{\mathcal{B}}}
\def \vcalD{\vec{\mathcal{D}}}
\def \vcalE{\vec{\mathcal{E}}}
\def \vcalP{\vec{\mathcal{P}}}
\def \vcalH{\vec{\mathcal{H}}}


% ----------- W
% ----------- X
\def \xhat{\hat{x}}
\def \Xhat{\hat{X}}
\def \xtilde{\tilde{x}}


% ----------- Y
% ----------- Z









% % ============================================================== NEWCOMMANDS
% ----------- A
% ----------- B
\newcommand \Bpow[1]{\bigg|#1\bigg|^2}
\newcommand \Bmean[1]{\bigg\langle #1\bigg\rangle}

% ----------- C
% ----------- D
\newcommand \Div[1]{\text{div} \l(#1\r)}
\newcommand \DKL[2]{\mathrm{D_{KL}}\bigg[#1\bigg|\bigg| #2\bigg]}
\newcommand \dkl[2]{\mathrm{D_{KL}}\l[#1\parallel #2\r]}
\newcommand \mydot[1]{\overset{\raisebox{0.1ex}{\scalebox{0.5}{$\bullet$}}}{#1}}

% ----------- E
\newcommand \Equiv[1]{\underset{#1}{\thicksim}}
\newcommand \Exp[1]{\cdot 10^{#1}}

% ----------- F
% ----------- G
\newcommand \grad[1]{\textbf{\text{grad}} \l(#1\r) }

% ----------- H
% ----------- I
\newcommand \integers[2]{\llbracket #1,#2 \rrbracket}


% ----------- J
% ----------- K
% ----------- L
\newcommand \logt[1]{\log\sub{10}\left(#1\right)}
\newcommand \lap[1]{\Delta \l(#1\r) }
\newcommand \lapvec[1]{\bm{\Delta} \l(#1\r) }

% ----------- M
\newcommand \mean[1]{\braket{#1}}
\newcommand \mymax[2]{\underset{#1}{\text{max}}\left[#2\right]}
\newcommand \myargmax[2]{\underset{#1}{\text{argmax}}\left[#2\right]}
\newcommand \myeq[1]{\underset{#1}{=}}
\newcommand \myunit[1]{~\mathrm{\left[#1\right]}}
\newcommand \mymatrix[1]{
    \begin{pmatrix}
        #1
    \end{pmatrix}
}

\newcommand \diag[1]{
    \mathbf{diag}\begin{pmatrix}
        #1
    \end{pmatrix}
}



% ----------- N\newcommand \grad[1]{\textbf{\text{grad}} \l(#1\r) }
\newcommand{\norm}[1]{\left\lVert #1 \right\rVert}


% ----------- O
% ----------- P
\newcommand \pow[1]{|#1|^2}
\newcommand \Partial[2]{\dfrac{\partial #1}{\partial #2}}
\newcommand \PartialHigher[3]{\dfrac{\partial^{#3} #1}{\partial #2^{#3}}}

% ----------- Q
% ----------- R
\newcommand \rot[1]{\textbf{\text{rot}} \l(#1\r)}

% ----------- S
\newcommand \std[1]{\sigma(#1)}
\newcommand \sub[1]{_{\mathrm{#1}}}


% ----------- T
\newcommand \To[1]{\underset{#1}{\to}}

% ----------- U
\newcommand \ups[1]{^{\mathrm{#1}}}


% ----------- V
\newcommand \var[1]{\sigma^2(#1)}
\newcommand \vlap[1]{\bm{\Delta} \l(\bm{#1}\r)}
\newcommand \vgrad[1]{\overrightarrow{\text{grad}} \l(#1\r) }

% ----------- W
% ----------- X
% ----------- Y
\def \ya{y_{\alpha}}
\def \yaI{y_{\alpha,I}}
\def \yaQ{y_{\alpha,Q}}
\def \yHI{\tilde{y}_{H,I}}
\def \yHQ{\tilde{y}_{H,Q}}
\def \ytilde{\tilde{y}}
\def \yVI{\tilde{y}_{V,I}}
\def \yVQ{\tilde{y}_{V,Q}}

% ----------- Z










































