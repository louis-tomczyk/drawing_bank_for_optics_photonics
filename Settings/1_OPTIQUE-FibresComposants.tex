% ------------------------------------------------------------------------------
% ----- INFORMATIONS -----
%   Author          : louis tomczyk, chatgpt
%   Institution     : Telecom Paris
%   Email           : louis.tomczyk@telecom-paris.fr
%   Arxivs          : 2024-02-11
%                   : 2025-01-02 [NEW] laser size, pd
%                   : 2025-01-03 [NEW] multiplexer
%                   : 2025-01-04 [NEW] poincare sphere
%   Date            : 2025-04-27 [NEW] poincare sphere tropique
%   Version         : 1.4.0
%   License         : cc-by-nc-sa
%                       CAN:    modify - distribute
%                       CANNOT: commercial use
%                       MUST:   share alike - include license
% ------------------------------------------------------------------------------
 
%%%%%%%%%%%%%%%%%%%%%%%%%%%%%%%%%%%%%%%%%%%%%%%%%%%%%%%%%%%%%%%%%%%%%%%%%%%%%%%%
\newcommand{\EDFA}[4]% #1 : abscisse, #2 ordonnée, #3 angle de rotation, #4 comment
{
    \draw[rotate around= {#3:(#1,#2)}, fill = white,thick] (#1,#2-0.5) --++ (1,0.5) --++ (-1,0.5) -- cycle;
    \draw[rotate around= {#3:(#1,#2)}] (#1+0.5,#2+0.75)node{\footnotesize #4};
}

%%%%%%%%%%%%%%%%%%%%%%%%%%%%%%%%%%%%%%%%%%%%%%%%%%%%%%%%%%%%%%%%%%%%%%%%%%%%%%%%
\newcommand{\FIBRE}[4]% #1 : abscisse, #2 ordonnée, #3 angle de rotation, #4 comment
{
    \draw[rotate around= {#3:(#1,#2)}, fill = white,thick] (#1,#2+0.5) circle (0.5) node[above = 7pt]{\footnotesize #4};

}

%%%%%%%%%%%%%%%%%%%%%%%%%%%%%%%%%%%%%%%%%%%%%%%%%%%%%%%%%%%%%%%%%%%%%%%%%%%%%%%%
\newcommand{\VOA}[4]% #1 : abscisse, #2 ordonnée, #3 angle de rotation, #4 comment
{
    \draw[thick] (#1-0.5,#2) --++ (1,0);
    \draw[rotate around= {#3:(#1,#2)}, fill = white,thick] (#1,#2) circle (0.25) node[above = 7pt]{\footnotesize #4};
    \draw[thick,->] (#1-0.3,#2-0.3) --++ (0.6,0.6);%
}

%%%%%%%%%%%%%%%%%%%%%%%%%%%%%%%%%%%%%%%%%%%%%%%%%%%%%%%%%%%%%%%%%%%%%%%%%%%%%%%%
\newcommand{\MZM}[4]% #1 : abscisse, #2 ordonnée, #3 supp length, #4 color
{
	\draw[color = #4,thick] (#1,#2) --++ (1,0) --++ (1,1) --++ (2,0) --++ (1,-1) --++ (1+#3,0) --++ (-1-#3,0) --++ (-1,-1) --++ (-2,0) --++ (-1,1);
}

%%%%%%%%%%%%%%%%%%%%%%%%%%%%%%%%%%%%%%%%%%%%%%%%%%%%%%%%%%%%%%%%%%%%%%%%%%%%%%%%
\newcommand{\MZMpi}[3]% #1 : abscisse, #2 ordonnée, #3 color
{
	\draw[color = #3,thick] (#1,#2) --++ (1,0) --++ (1,1) --++ (2,0) --++ (1,-1) --++ (1,0) --++ (-1,0) --++ (-1,-1) --++ (-2,0) --++ (-1,1);
	\draw[color = #3,fill = white,thick] (#1+5.5,#2-0.5) rectangle (#1+6.5,#2+0.5) node[midway]{$\dfrac{\pi}{2}$};
	\draw[color = #3] (#1+6.5,#2) --++ (0.5,0);
}

%%%%%%%%%%%%%%%%%%%%%%%%%%%%%%%%%%%%%%%%%%%%%%%%%%%%%%%%%%%%%%%%%%%%%%%%%%%%%%%%
\newcommand{\VoltBox}[4]% #1 : abscisse, #2 ordonnée, #3 text, #4 color edges
{
	\draw[color = #4] (#1,#2) rectangle (#1+1,#2+1) node[midway]{$U#3$};
	\draw[color = #4,->,thick] (#1+0.5,#2+1) --++ (0,0.5);
	\draw[color = #4,->,thick] (#1+0.5,#2) --++ (0,-0.5);
}

%%%%%%%%%%%%%%%%%%%%%%%%%%%%%%%%%%%%%%%%%%%%%%%%%%%%%%%%%%%%%%%%%%%%%%%%%%%%%%%%
\newcommand{\PBS}[3]% #1 : abscisse, #2 ordonnée, #3 scale
{
    \draw[fill = white] (#1*#3,#2*#3) rectangle (#1*#3+1*#3,#2*#3+1*#3) node[midway, below = 5pt]{\footnotesize PBS};
\draw (#1*#3,#2*#3) --++ (1*#3,0) --++ (0,1*#3) --++ (-1*#3,-1*#3) --++ (0,1*#3) --++ (1*#3,0);
	\draw (#1*#3+0.75*#3,#2*#3+0.1875*#3) node{$\bullet$};
}

%%%%%%%%%%%%%%%%%%%%%%%%%%%%%%%%%%%%%%%%%%%%%%%%%%%%%%%%%%%%%%%%%%%%%%%%%%%%%%%%
\newcommand{\LED}[3] % #1 abscisse, #2 ordonnée, #3 color
{
	\draw[thick, color = #3] (#1,#2+1.5) --++ (0,-0.75);
	\draw[thick, color = #3] (#1-0.75,#2-0.75) rectangle (#1+0.75,#2+0.75);
	\draw[thick, color = #3] (#1,#2+0.625) --++ (0,-0.25) --++ (-0.4,0) --++ (0.4,-0.625) --++ (0,-0.25) --++ (0,0.25) --++ (-0.375,0) --++ (0.75,0) --++ (-0.375,0)--++ (0.4,0.625) --++ (-0.4,0);
	
	\draw[thick, color = #3] (#1+0.4,#2-0.5) node[rotate = -40]{$\to$};
	\draw[thick, color = #3] (#1+0.2,#2-0.52) node[rotate = -90]{$\to$};
	
	\draw[thick, color = #3] (#1,#2-0.75) --++ (0,-0.75);
}

%%%%%%%%%%%%%%%%%%%%%%%%%%%%%%%%%%%%%%%%%%%%%%%%%%%%%%%%%%%%%%%%%%%%%%%%%%%%%%%%
\newcommand{\PD}[4] % #1 abscisse, #2 ordonnée, #3 color
{
	\draw[thick, color = #3] (#1*#4-0.75*#4,#2*#4-0.75*#4) rectangle (#1*#4+0.75*#4,#2*#4+0.75*#4);
	\draw[thick, color = #3] (#1*#4,#2*#4+0.625*#4) --++ (0,-0.25*#4) --++ (-0.4*#4,0) --++ (0.4*#4,-0.625*#4) --++ (0,-0.25*#4) --++ (0,0.25*#4) --++ (-0.375*#4,0) --++ (0.75*#4,0) --++ (-0.375*#4,0)--++ (0.4*#4,0.625*#4) --++ (-0.4*#4,0);
	
	\draw[thick, color = #3] (#1*#4-0.45*#4,#2*#4-0.5*#4) node[rotate = 40]{\scriptsize $\bm{\to}$};
	\draw[thick, color = #3] (#1*#4-0.25*#4,#2*#4-0.52*#4) node[rotate = 40]{\scriptsize$\bm{\to}$};
}

%%%%%%%%%%%%%%%%%%%%%%%%%%%%%%%%%%%%%%%%%%%%%%%%%%%%%%%%%%%%%%%%%%%%%%%%%%%%%%%%
\newcommand{\laser}[3] % #1 abscisse, #2 ordonnée, #3 color
{
	\draw[thick, color = #3] (#1,#2+1.5) --++ (0,-0.75);
	\draw[thick, color = #3] (#1-0.75,#2-0.75) rectangle (#1+0.75,#2+0.75);
	\draw[thick, color = #3] (#1,#2+0.625) --++ (0,-0.25) --++ (-0.4,0) --++ (0.4,-0.625) --++ (0,-0.25) --++ (0,0.25) --++ (-0.375,0) --++ (0.75,0) --++ (-0.375,0)--++ (0.4,0.625) --++ (-0.4,0);
	\draw[thick, color = #3] (#1+0.3,#2-0.5) node[rotate = -40]{$\to$};
	\draw[thick, color = #3] (#1+0.5,#2-0.5) node[rotate = -45]{$\to$};
	\draw[thick, color = #3] (#1,#2-0.75) --++ (0,-0.75);
%	\draw[snake = coil, segment aspect  = 0, segment length = 2 mm, color = #3] (#1-0.25,#2+0.25) --++ (0.5,0);
}

%%%%%%%%%%%%%%%%%%%%%%%%%%%%%%%%%%%%%%%%%%%%%%%%%%%%%%%%%%%%%%%%%%%%%%%%%%%%%%%%
\newcommand{\laserSize}[5] % #1 abscisse, #2 ordonnée, #3 color, #4 scale, #5 comment
{
	\draw[color = black] (#1*#4-0.75*#4,#2*#4-0.75*#4) rectangle (#1*#4+0.75*#4,#2*#4+0.75*#4)node[midway,left = 0.5cm]{#5};
	\draw[thick, color = #3, fill = #3] (#1*#4,#2*#4+0.625*#4) --++ (0,-0.25*#4) --++ (-0.4*#4,0) --++ (0.4*#4,-0.625*#4) --++ (0,-0.25*#4) --++ (0,0.25*#4) --++ (-0.375*#4,0) --++ (0.75*#4,0) --++ (-0.375*#4,0)--++ (0.4*#4,0.625*#4) --++ (-0.4*#4,0);
	\draw[thick, color = #3] (#1*#4+0.3*#4,#2*#4-0.5*#4) node[rotate = -40]{\footnotesize $\bm{\to}$};
	\draw[thick, color = #3] (#1*#4+0.5*#4,#2*#4-0.5*#4) node[rotate = -45]{\footnotesize $\bm{\to}$};
}

%%%%%%%%%%%%%%%%%%%%%%%%%%%%%%%%%%%%%%%%%%%%%%%%%%%%%%%%%%%%%%%%%%%%%%%%%%%%%%%%
\newcommand{\MULTIPLEXER}[4]{%
  % #1 : Abscisse du "centre" du trapèze
  % #2 : Ordonnée du "centre" du trapèze
  % #3 : Facteur d'échelle (scale)
  % #4 : Angle de rotation (en degrés)
  \begin{scope}[
    shift={({#1-0.25},{#2-1.5})},  % Décalage pour "placer" le centre
    scale={#3},                    % Échelle
    rotate={#4}                    % Rotation autour du nouveau centre
  ]
    %%%%%%%%%%%%%%%%%%%%%%%%%%%%%%%
    %%%  CANAUX MULTIPLEXES   %%%%%
    %%%%%%%%%%%%%%%%%%%%%%%%%%%%%%%
    \draw[rounded corners=5pt, thick, draw=red]
        (0.5,2.5) --++ (1,0) --++ (0.25,-1) --++ (0.25,0 );
    \draw[dashed, thick, red]
        (-0.5, 2.65) -- (0.5,2.5);

    \draw[rounded corners=5pt, thick, draw=orange]
        (0.5,2.25) --++ (1,0) --++ (0.25,-0.75) --++ (0.25,0 );
    \draw[dashed, thick, orange]
        (-0.5, 2.35) -- (0.5,2.25);

    \draw[rounded corners=5pt, thick, draw=yellow]
        (0.5,2) --++ (1,0) --++ (0.25,-0.5) --++ (0.25,0 );
    \draw[dashed, thick, yellow]
        (-0.5, 2.1) -- (0.5,2);

    \draw[rounded corners=5pt, thick, draw=green]
        (0.5,1.75) --++ (1,0) --++ (0.25,-0.25) --++ (0.25,0 );
    \draw[dashed, thick, green]
        (-0.5, 1.85) -- (0.5,1.75);

    \draw[rounded corners=5pt, thick, draw=cyan]
        (0.5,1.25) --++ (1,0) --++ (0.25,0.25) --++ (0.25,0 );
    \draw[dashed, thick, cyan]
        (-0.5, 1.15) -- (0.5,1.25);

    \draw[rounded corners=5pt, thick, draw=blue]
        (0.5,1) --++ (1,0) --++ (0.25,0.5) --++ (0.25,0 );
    \draw[dashed, thick, blue]
        (-0.5, 0.9) -- (0.5,1);

    \draw[rounded corners=5pt, thick, draw=purple]
        (0.5,0.75) --++ (1,0) --++ (0.25,0.75) --++ (0.25,0 );
    \draw[dashed, thick, purple]
        (-0.5, 0.65) -- (0.5,0.75);

    \draw[rounded corners=5pt, thick, draw=purple!50!black]
        (0.5,0.5) --++ (1,0) --++ (0.25,1) --++ (0.25,0 );
    \draw[dashed, thick, purple!50!black]
        (-0.5, 0.4) -- (0.5,0.5);

    %%%%%%%%%%%%%%%%%%%%%%%%%%%%%%%
    %%%  FIBRE ET TRAPÈZE   %%%%%%%
    %%%%%%%%%%%%%%%%%%%%%%%%%%%%%%%
    \draw[thick] (-0.5,1.5) --++ (2.5,0);

    \draw[fill=white] 
      (0,3)        coordinate(A)
      -- (0.5,2.5) coordinate(B)
      -- (0.5,0.5) coordinate(C)
      -- (0,0)     coordinate(D)
      -- cycle;
      
    \path let 
      % On calcule le centre A-B
      \p1 = ($ (A)!0.5!(B) $),
      % Le centre C-D
      \p2 = ($ (C)!0.5!(D) $)
    in coordinate (trapCenter) at ($ (\p1)!0.5!(\p2) $);
    
    % On place le texte tourné autour de ce point
    \node[rotate=#4-90] at (trapCenter) {\small multiplexeur};

  \end{scope}
}


%%%%%%%%%%%%%%%%%%%%%%%%%%%%%%%%%%%%%%%%%%%%%%%%%%%%%%%%%%%%%%%%%%%%%%%%%%%%%%%%
\newcommand{\SpherePoincareAxes}[3]{%
  \begin{tikzpicture}[
    shift={(#1,#2)},
    scale=#3,
    line cap=round,
    line join=round
  ]

  % --- Cercle "extérieur" (projection de l'équateur) ---
  \draw[thick] (1,0) arc[start angle=0, end angle=360, radius=1];

  % --- Ellipse (arrière pointillé + avant plein) ---
  \draw[dashed,lightgray] (1,0)
       arc[start angle=0, end angle=180, x radius=1cm, y radius=0.4cm];
  \draw[thick] (-1,0)
       arc[start angle=180, end angle=360, x radius=1cm, y radius=0.4cm];

  % --- Axes S1, S2, S3 (en pointillés jusqu'à un point, puis flèche) ---
  \draw[gray, dotted] (0,0,0) --++ (0,0,1);
  \draw[gray, dotted] (0,0,0) --++ (1,0,0);
  \draw[gray, dotted] (0,0,0) --++ (0,1,0);

  \draw[thick, ->] (0,0,1) --++ (0,0,0.2) node[below left]  {$S_1$};
  \draw[thick, ->] (1,0,0) --++ (0.3,0,0) node[right]       {$S_2$};
  \draw[thick, ->] (0,1,0) --++ (0,0.3,0) node[above left]  {$S_3$};


  %%%%%%%%%%%%%%%%%%%%%%%%%%%%%%%%%%%%%%%%%%%%%%%%%%%%%%%%
  %   Flèche "état de polarisation" quelconque en 3D
  %%%%%%%%%%%%%%%%%%%%%%%%%%%%%%%%%%%%%%%%%%%%%%%%%%%%%%%%

  \pgfmathsetmacro{\Theta}{75} % Angle polaire (par rapport à S1=z)
  \pgfmathsetmacro{\phi}{35}   % Angle azimutal (dans le plan x-y => S2,S3)

  % Conversion sphérique -> cartésienne
  \pgfmathsetmacro{\x}{cos(\phi)*sin(\Theta)}
  \pgfmathsetmacro{\y}{sin(\phi)*sin(\Theta)}
  \pgfmathsetmacro{\z}{cos(\Theta)}

  % Flèche épaisse vers (x,y,z)
  \draw[ultra thick, ->, orange] (0,0,0) -- (\x,\y,\z);

  \draw[->] (\x,\y,\z) --++ (0.6,0.48,0.6) node[right]{$\vecu_S$};
  \draw[->] (\x,\y,\z) --++ (-0.3,0.4,-0.1) node[left, above]{$\vecu_\chi$};
  \draw[->] (\x,\y,\z) --++ (0.1,0.1,-0.3) node[left, above]{$\vecu_\theta$};

  % Projection en pointillés (sur S1-S3 => x=0)
  \draw[dashed,orange] (0,0,0) -- (\x,0,\z);

  % Arcs pour 2chi et 2theta
  \draw[->,orange] (\x,0,\z) arc (0:28:1) node[midway, left = 3pt] {$\bm{2\chi}$};
  \draw[->,orange] (0,0,0.6) arc[start angle=-55, end angle=0,
            x radius=1.1, y radius=0.25] node[midway, below = 3pt] {$\bm{2\theta}$};

  \end{tikzpicture}
}


%%%%%%%%%%%%%%%%%%%%%%%%%%%%%%%%%%%%%%%%%%%%%%%%%%%%%%%%%%%%%%%%%%%%%%%%%%%%%%%%
\newcommand{\SpherePoincare}[3]{%
\begin{tikzpicture}[
shift={(#1,#2)},
scale=#3,
line cap=round,
line join=round
]

% --- Cercle "extérieur" (projection de l'équateur) ---
\draw[thick] (1,0) arc[start angle=0, end angle=360, radius=1];

% --- Ellipse (arrière pointillé + avant plein) ---
\draw[dashed,lightgray] (1,0) arc[start angle=0, end angle=180, x radius=1cm, y radius=0.4cm];
\draw[thick] (-1,0) arc[start angle=180, end angle=360, x radius=1cm, y radius=0.4cm];

% --- Axes S1, S2, S3 (en pointillés jusqu'à un point, puis flèche) ---
\draw[gray, dotted] (0,0,0) --++ (0,0,1);
\draw[gray, dotted] (0,0,0) --++ (1,0,0);
\draw[gray, dotted] (0,0,0) --++ (0,1,0);

\draw[thick, ->] (0,0,1) --++ (0,0,0.2) node[below left]  {$S_1$};
\draw[thick, ->] (1,0,0) --++ (0.3,0,0) node[right]       {$S_2$};
\draw[thick, ->] (0,1,0) --++ (0,0.3,0) node[above left]  {$S_3$};

% S2 == x, 	S3 == y, 	S1 == z
% 2theta == theta, 	2chi == phi
\pgfmathsetmacro{\Theta}{75} % Angle polaire (par rapport à S1=z)
\pgfmathsetmacro{\phi}{35}   % Angle azimutal (dans le plan x-y => S2,S3)

% Conversion sphérique -> cartésienne
\pgfmathsetmacro{\x}{cos(\phi)*sin(\Theta)}
\pgfmathsetmacro{\y}{sin(\phi)*sin(\Theta)}
\pgfmathsetmacro{\z}{cos(\Theta)}

% Flèche épaisse vers (x,y,z)
\draw[thick,dashed, ->, orange] (0,0,0) -- (\x,\y,\z);
\draw[->,thick,orange] (\x,\y,\z) --++ (0.6,0.48,0.6) node[right]{$\vecu_\vecS$};
\draw[->,thick,orange] (\x,\y,\z) --++ (-0.3,0.4,-0.1) node[left, above]{$\vecu_\chi$};
\draw[->,thick,orange] (\x,\y,\z) --++ (0.1,0.1,-0.3) node[left, above]{$\vecu_\theta$};

% Projection en pointillés (sur S1-S3 => x=0)
\draw[dashed,orange] (0,0,0) -- (\x,0,\z);

% Arcs pour 2chi et 2theta
\draw[->,orange] (\x,0,\z) arc (0:28:1) node[midway, left] {$\bm{2\chi}$};
\draw[->,orange] (0,0,0.6) arc[start angle=-55, end angle=0,
        x radius=1.1, y radius=0.25] node[midway, below] {$\bm{2\theta}$};

\end{tikzpicture}
}


%%%%%%%%%%%%%%%%%%%%%%%%%%%%%%%%%%%%%%%%%%%%%%%%%%%%%%%%%%%%%%%%%%%%%%%%%%%%%%%%
\newcommand{\SpherePoincareTrop}[3]{%
	\begin{tikzpicture}[
	shift={(#1,#2)},
	scale=#3,
	line cap=round,
	line join=round
	]

	% --- Cercle extérieur (équateur) ---
	\draw[thick] (1,0) arc[start angle=0, end angle=360, radius=1];

	% --- Ellipse (arrière pointillé + avant plein) ---
	\draw[dashed,lightgray] (1,0)
	   arc[start angle=0, end angle=180, x radius=1cm, y radius=0.4cm];
	\draw[thick] (-1,0)
	   arc[start angle=180, end angle=360, x radius=1cm, y radius=0.4cm];

	% --- Axes S1, S2, S3 ---
	\draw[gray, dotted] (0,0,0) --++ (0,0,1);
	\draw[gray, dotted] (0,0,0) --++ (1,0,0);
	\draw[gray, dotted] (0,0,0) --++ (0,1,0);

	\draw[thick, ->] (0,0,1) --++ (0,0,0.2) node[below left]  {$S_1$};
	\draw[thick, ->] (1,0,0) --++ (0.3,0,0) node[right]       {$S_2$};
	\draw[thick, ->] (0,1,0) --++ (0,0.3,0) node[above left]  {$S_3$};

	%%%%%%%%%%%%%%%%%%%%%%%%%%%%%%%%%%%%%%%%%%%%%%%%%%%%%%%%
	%   État de polarisation (original)
	%%%%%%%%%%%%%%%%%%%%%%%%%%%%%%%%%%%%%%%%%%%%%%%%%%%%%%%%
	% S2 == x, 	S3 == y, 	S1 == z
	% 2theta == theta, 	2chi == phi
	\pgfmathsetmacro{\Theta}{65} 
	\pgfmathsetmacro{\phi}{34}   
	\pgfmathsetmacro{\x}{cos(\phi)*sin(\Theta)}
	\pgfmathsetmacro{\y}{sin(\phi)*sin(\Theta)}
	\pgfmathsetmacro{\z}{cos(\Theta)}

	\draw[thick,dashed, ->, orange] (0,0,0) -- (\x,\y,\z);
	\draw[dashed,orange] (0,0,0) -- (\x,0,\z);
	\draw[->,orange] (\x,0,\z) arc (0:27:1) node[midway, left] {$\bm{2\chi}$};
	\draw[<-,orange] (0.5*\x,0,0.5*\z) arc[start angle=0, end angle=-85,
		    x radius=0.55, y radius=0.125] node[xshift = -.65cm] {$\bm{2\theta = \Blue{\xi}}$};

	%%%%%%%%%%%%%%%%%%%%%%%%%%%%%%%%%%%%%%%%%%%%%%%%%%%%%%%%
	%   Tropique (parallèle nord) - Ajout
	%%%%%%%%%%%%%%%%%%%%%%%%%%%%%%%%%%%%%%%%%%%%%%%%%%%%%%%%
	\pgfmathsetmacro{\TropiqueAngle}{30} % Angle du tropique depuis S1 (z)
	\pgfmathsetmacro{\TropiqueRadius}{sin(\TropiqueAngle)} % Rayon projeté
	\pgfmathsetmacro{\TropiqueHeight}{cos(\TropiqueAngle)} % Hauteur sur S1

	\draw[thick, dashed, blue] (0.848,0.5)
	   arc[start angle=0, end angle=180, x radius=0.85cm, y radius=0.23cm];
	\draw[ultra thick, blue] (0.848,0.5)
	   arc[start angle=0, end angle=-180, x radius=0.85cm, y radius=0.23cm];
       
\end{tikzpicture}
}



%%%%%%%%%%%%%%%%%%%%%%%%%%%%%%%%%%%%%%%%%%%%%%%%%%%%%%%%%%%%%%%%%%%%%%%%%%%%%%%%
\newcommand{\SpherePoincareXi}[3]{%
  \begin{tikzpicture}[
    shift={(#1,#2)},
    scale=#3,
    line cap=round,
    line join=round
  ]

% Paramètres de l'état
\def\Theta{65}
\def\phi{34}
\pgfmathsetmacro{\x}{cos(\phi)*sin(\Theta)}
\pgfmathsetmacro{\y}{sin(\phi)*sin(\Theta)}
\pgfmathsetmacro{\z}{cos(\Theta)}

% Sphère de Poincaré 3D
\draw[thick] (1,0,0) arc (0:360:1);
\draw[dashed,gray] (1,0,0) arc (0:180:1 and 0.4);
\draw[thick] (-1,0,0) arc (180:360:1 and 0.4);

% Axes 3D

\draw[gray, dotted] (0,0,0) --++ (0,0,1);
\draw[gray, dotted] (0,0,0) --++ (1,0,0);
\draw[gray, dotted] (0,0,0) --++ (0,1,0);

\draw[thick, ->] (0,0,1) --++ (0,0,0.2) node[below left]  {$S_1$};
\draw[thick, ->] (1,0,0) --++ (0.3,0,0) node[right]       {$S_2$};
\draw[thick, ->] (0,1,0) --++ (0,0.3,0) node[above left]  {$S_3$};

% Vecteur d'axe (avec votre décalage original)
\draw[orange,->,dashed] (0,0,0) -- (\x+0.08,\y+0.24,\z);
\draw[dashed,orange] (0,0,0) -- (\x,0,\z);
\draw[orange,->,thick] (\x+0.08,\y+0.24,\z) --++ (0.2,0.2,0) node[above right] {$\vecu_\vecS$};
\draw[->,orange,thick] (\x,0,\z) arc (0:32:1) node[midway, below left] {$2\chi$};
\draw[<-,orange] (0.5*\x,0,0.5*\z) arc[start angle=0, end angle=-85,
	    x radius=0.55, y radius=0.125] node[xshift = -.65cm] {$2\theta \neq \Blue{\xi}$};

\draw[blue, ultra thick,rotate around={45:(\x/2,\y/2,\z/2)}] (0.9*\x+0.015,0.8*\y+0.8,0.9*\z) arc (90:270:0.2 and 0.8);
\draw[blue,thick,dashed, rotate around={45:(\x/2,\y/2,\z/2)}] (0.9*\x+0.015,0.8*\y+0.8,0.9*\z) arc (90:-270:0.2 and 0.8);

\draw[dotted,thick,blue] (\x-0.22,\y,\z) --++ (-0.1,0.45,0);
\draw[dotted,thick,blue] (\x-0.22,\y,\z) --++ (-0.55,0.6,0);
\draw[->,blue,thick] (\x-0.22-0.078,\y+0.35,\z) arc (0:140:0.15 and 0.07)
node[midway,below]{$\bm{\xi}$};

    
\end{tikzpicture}
}




%%%%%%%%%%%%%%%%%%%%%%%%%%%%%%%%%%%%%%%%%%%%%%%%%%%%%%%%%%%%%%%%%%%%%%%%%%%%%%%%
\newcommand{\SpherePoincareVsop}[3]{%
\begin{tikzpicture}[
shift={(#1,#2)},
scale=#3,
line cap=round,
line join=round
]

% --- Cercle extérieur (équateur) ---
\draw[thick] (1,0) arc[start angle=0, end angle=360, radius=1];

% --- Ellipse (arrière pointillé + avant plein) ---
\draw[dashed,gray] (1,0)
   arc[start angle=0, end angle=180, x radius=1cm, y radius=0.4cm];
\draw[thick] (-1,0)
   arc[start angle=180, end angle=360, x radius=1cm, y radius=0.4cm];

% --- Axes S1, S2, S3 ---
\draw[gray, dotted] (0,0,0) --++ (0,0,1);
\draw[gray, dotted] (0,0,0) --++ (1,0,0);
\draw[gray, dotted] (0,0,0) --++ (0,1,0);

\draw[thick, ->] (0,0,1) --++ (0,0,0.2) node[below left]  {$S_1$};
\draw[thick, ->] (1,0,0) --++ (0.3,0,0) node[right]       {$S_2$};
\draw[thick, ->] (0,1,0) --++ (0,0.3,0) node[above left]  {$S_3$};

\pgfmathsetmacro{\ThetaA}{75} 
\pgfmathsetmacro{\phiA}{34}   
\pgfmathsetmacro{\xA}{cos(\phiA)*sin(\ThetaA)}
\pgfmathsetmacro{\yA}{sin(\phiA)*sin(\ThetaA)}
\pgfmathsetmacro{\zA}{cos(\ThetaA)}

\pgfmathsetmacro{\ThetaB}{85} 
\pgfmathsetmacro{\phiB}{80}   
\pgfmathsetmacro{\xB}{cos(\phiB)*sin(\ThetaB)}
\pgfmathsetmacro{\yB}{sin(\phiB)*sin(\ThetaB)}
\pgfmathsetmacro{\zB}{cos(\ThetaB)}

\pgfmathsetmacro{\ThetaC}{80} 
\pgfmathsetmacro{\phiC}{57}   
\pgfmathsetmacro{\xC}{cos(\phiC)*sin(\ThetaC)}
\pgfmathsetmacro{\yC}{sin(\phiC)*sin(\ThetaC)}
\pgfmathsetmacro{\zC}{cos(\ThetaC)}

\draw[thick, dashed, ->, orange] (0,0,0)node[below]{\textcolor{black}{$O$}} -- (\xA,\yA,\zA) node[midway, right = 3pt]{$\vecS_t$};
\draw[thick, dashed, ->, blue] (0,0,0) -- (\xB,\yB,\zB) node[midway, left]{$\vecS_{t+dt}$};
\draw[<->,thick] (\xA,\yA,\zA) -- (\xB,\yB,\zB) node[sloped, midway, below = 1pt]{$||\Delta \vecS_t||$};

\draw[<->, thick] (\xA+0.08,\yA+0.08,\zA+0.08) arc (0:90:0.52) node[midway, above right] {$\Delta\alpha_t$};
\end{tikzpicture}
}

