% ------------------------------------------------------------------------------
% ----- INFORMATIONS -----
%   Author          : louis tomczyk
%   Institution     : Telecom Paris
%   Email           : louis.tomczyk@telecom-paris.fr
%   Arxivs          :
%   Date            : 2024-02-19
%   Version         : 1.0.0
%   License         : cc-by-nc-sa
%                       CAN:    modify - distribute
%                       CANNOT: commercial use
%                       MUST:   share alike - include license
% ------------------------------------------------------------------------------

%%%%%%%%%%%%%%%%%%%%%%%%%%%%%%%%%%%%%%%%%%%%%%%%%%%%%%%%%%%%%%%%%%%
% CHARTE GRAPHIQUE TELECOM
%%%%%%%%%%%%%%%%%%%%%%%%%%%%%%%%%%%%%%%%%%%%%%%%%%%%%%%%%%%%%%%%%%%

\newcommand{\rectTelecom}[3]{
    \draw[color = #1, fill = #1] ($(#2)$) rectangle ($(#2)+(#3,#3/11.6)$);
}


%%%%%%%%%%%%%%%%%%%%%%%%%%%%%%%%%%%%%%%%%%%%%%%%%%%%%%%%%%%%%%%%%%%
% ORGANIGRAMMES
%%%%%%%%%%%%%%%%%%%%%%%%%%%%%%%%%%%%%%%%%%%%%%%%%%%%%%%%%%%%%%%%%%%

% boxes
% #1 : abscisse start, #2 ordonnée start, #3 abscisse end, #4 ordonnée end, #5 color, #6 fill , #7 opacity, #8 comment
\newcommand{\mybox}[8]
{
    \draw[rounded corners = 4pt,thick,color = #5,fill = #6,fill opacity= #7]
    (#1,#2) rectangle (#1+#3,#2+#4)
    node[text = black,text opacity = 1,midway]{#8};
}

\newcommand{\myboxopaque}[6]% #1 : abscisse, #2 ordonnée, #3 comment
{
    \draw[rounded corners = 4pt,thick,color = #5,fill = #6]
    (#1,#2) rectangle (#1+#3,#2+#4);
}

\newcommand{\boxvalue}[4]% #1 : abscisse, #2 ordonnée, #3 comment
{
    \draw[rounded corners = 4pt,thick,color = #4,fill = lightgray,fill opacity= 0.2]
    (#1-1,#2-0.5) rectangle (#1+1,#2+0.5)
    node[text = black,text opacity = 1,midway]{#3};
}

\newcommand{\boxvaluesmaller}[4]% #1 : abscisse, #2 ordonnée, #3 comment
{
    \draw[rounded corners = 4pt,thick, color = #4,fill = lightgray,fill opacity= 0.2]
    (#1-0.75,#2-0.5) rectangle (#1+0.75,#2+0.5)
    node[text = black,text opacity = 1,midway]{#3};
}

\newcommand{\boxvaluelarger}[4]% #1 : abscisse, #2 ordonnée, #3 comment
{
    \draw[rounded corners = 4pt,thick, color = #4,fill = lightgray,fill opacity= 0.2]
    (#1-1.25,#2-0.5) rectangle (#1+1.25,#2+0.5)
    node[text = black,text opacity = 1,midway]{#3};
}

\newcommand{\boxvaluelarge}[3]% #1 : abscisse, #2 ordonnée, #3 comment
{
    \draw[rounded corners = 4pt,thick,fill = cyan, color = cyan,fill opacity= 0.2]
    (#1-2,#2-0.5) rectangle (#1+2,#2+0.5)
    node[text = black,text opacity = 1,midway]{#3};
}

\newcommand{\boxvalueLargeCustom}[5]% #1 : abscisse, #2 ordonnée, #3 comment
{
    \draw[rounded corners = 4pt,thick, color = #4,fill = lightgray,fill opacity= 0.2]
    (#1-#5,#2-0.5) rectangle (#1+#5,#2+0.5)
    node[text = black,text opacity = 1,midway]{#3};
}

\newcommand{\boxaction}[4]% #1 : abscisse, #2 ordonnée, #3 comment, #4 color
{
    \draw[rounded corners = 4pt,thick, color = #4] (#1-1,#2-0.5) rectangle (#1+1,#2+0.5)
    node[midway]{#3};
}

\newcommand{\boxactionlarge}[3]% #1 : abscisse, #2 ordonnée, #3 comment
{
    \draw[rounded corners = 4pt,thick] (#1-2,#2-0.5) rectangle (#1+2,#2+0.5)
    node[midway]{#3};
}

\newcommand{\boxactionthick}[3]% #1 : abscisse, #2 ordonnée, #3 comment
{
    \draw[rounded corners = 4pt,thick] (#1-2,#2-1) rectangle (#1+2,#2+1)
    node[midway]{#3};
}

\newcommand{\boxquestion}[3]% #1 : abscisse, #2 ordonnée, #3 comment
{
    \draw[rounded corners = 4pt,thick,fill = green, color = green,fill opacity= 0.2] (#1-1,#2-0.5) rectangle (#1+1,#2+0.5)
    node[text = black,text opacity = 1,midway]{#3};
}

\newcommand{\boxquestionlarge}[3]% #1 : abscisse, #2 ordonnée, #3 comment
{
    \draw[rounded corners = 4pt,thick,fill = green, color = green,fill opacity= 0.2] (#1-2,#2-0.5) rectangle (#1+2,#2+0.5)
    node[text = black,text opacity = 1,midway]{#3};
}

\newcommand{\boxbreak}[5]% #1 : abscisse center, #2 ordonnée center, #3 demi petit axe, #4 demi grand axe, #5 comment
{
    \draw[rounded corners = 4pt,thick,fill = red, color = red,fill opacity= 0.35] (#1-2,#2-0.5) ellipse  (#3 cm and #4 cm)
    node[text = black,text opacity = 1]{#5};
}

\newcommand{\boxvalueopaque}[6]% #1 : abscisse, #2 ordonnée, #3 comment
{
    \draw[rounded corners = 4pt,thick,color = #5,fill = lightgray,fill opacity= 1]
    (#1,#2) rectangle (#1+#3,#2+#4)
    node[text = black,text opacity = 1,midway]{#6};
}

\newcommand{\comparator}[4]% #1 : abscisse, #2 ordonnée, #3 radius
{
    \draw[thick] (#1,#2) circle (#3) node{\textcolor{#4}{$+$}};
}

% links
%\newcommand{\linkright}[5]% #1 : abscisse, #2 ordonnée, #3 length, #4 comment, #5 color
%{    \draw[thick,->,color = #5] (#1,#2) --++ (#3,0) node[midway,above]{#4};}

\newcommand{\linkright}[5]% #1 : x, #2 : y, #3 : length, #4 : texte, #5 : couleur (optionnel)
{
    \def\linkcolor{#5}%
    \ifx\linkcolor\empty \def\linkcolor{black}\fi % Si #5 est vide, couleur = noir
    \draw[thick,->,color=\linkcolor] (#1,#2) --++ (#3,0) node[midway,above]{#4};
}

\newcommand{\linkdown}[5]% #1 : abscisse, #2 ordonnée, #3 length, #4 comment, #5 color
{
    \draw[thick,->, color = #5] (#1,#2) --++ (0,-#3) node[midway,right]{#4};
}

\newcommand{\linksloped}[5]% #1 : abscisse, #2 ordonnée, #3 length, #4 comment
{
    \draw[thick,->] (#1,#2) --++ (#3,#4) node[midway,right]{#5};
}
